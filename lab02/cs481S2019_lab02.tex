\documentclass[11pt]{article}

% NOTE: The "Edit" sections are changed for each assignment

% Edit these commands for each assignment

\newcommand{\assignmentduedate}{May 2}
\newcommand{\assignmentassignedate}{March 28}
\newcommand{\assignmentnumber}{Two}

\newcommand{\labassignmentnumberstart}{Two}
\newcommand{\labassignmentnumberend}{Two}

\newcommand{\labyear}{2019}
\newcommand{\labdueday}{Thursday}
\newcommand{\labassignday}{Thursday}
\newcommand{\labtime}{1:30 pm}

\newcommand{\assigneddate}{Assigned: \labassignday, \assignmentassignedate, \labyear{} at \labtime{}}
\newcommand{\duedate}{Due: \labdueday, \assignmentduedate, \labyear{} at \labtime{}}

% NOTE: there was no source code for this assignment
% Instead, the focus was on two Markdown files

\newcommand{\assessment}{\lstinline{assessment.md}}
\newcommand{\conduct}{\lstinline{conduct.md}}

% Edit these commands to give the name to the main program

\newcommand{\mainprogram}{\lstinline{compute_tf_monolith.py}}
\newcommand{\mainprogramsource}{\lstinline{src/termfrequency/compute_tf_monolith.py}}

% Edit this commands to describe key deliverables

\newcommand{\reflection}{\lstinline{report.md}}

% Commands to describe key development tasks

% --> Running gatorgrader.sh
\newcommand{\gatorgraderstart}{\command{gradle grade}}
\newcommand{\gatorgradercheck}{\command{gradle grade}}

% --> Compiling and running program with gradle
\newcommand{\gradlebuild}{\command{gradle build}}
\newcommand{\gradlerun}{\command{gradle run}}

% Commands to describe key git tasks

% NOTE: Could be improved, problems due to nesting

\newcommand{\gitcommitfile}[1]{\command{git commit #1}}
\newcommand{\gitaddfile}[1]{\command{git add #1}}

\newcommand{\gitadd}{\command{git add}}
\newcommand{\gitcommit}{\command{git commit}}
\newcommand{\gitpush}{\command{git push}}
\newcommand{\gitpull}{\command{git pull}}

\newcommand{\gitcommitmainprogram}{\command{git commit src/termfrequency/compute_tf_monolith.py -m "Descriptive commit message"}}

% Commands for the textbooks, since there are so many

\newcommand{\cooperative}{{\em Cooperative Software Design\/}}
\newcommand{\philosophy}{{\em Philosophy of Software Design\/}}
\newcommand{\thinkpython}{{\em Think Python\/}}
\newcommand{\programmingstyle}{{\em Exercises in Programming Style\/}}
\newcommand{\pytest}{{\em Python Testing with Pytest\/}}

% Use this when displaying a new command

\newcommand{\command}[1]{``\lstinline{#1}''}
\newcommand{\program}[1]{\lstinline{#1}}
\newcommand{\url}[1]{\lstinline{#1}}
\newcommand{\channel}[1]{\lstinline{#1}}
\newcommand{\option}[1]{``{#1}''}
\newcommand{\step}[1]{``{#1}''}

\usepackage{pifont}
\newcommand{\checkmark}{\ding{51}}
\newcommand{\naughtmark}{\ding{55}}

\usepackage{listings}
\lstset{
  basicstyle=\small\ttfamily,
  columns=flexible,
  breaklines=true
}

\usepackage{fancyhdr}

\usepackage[margin=1in]{geometry}
\usepackage{fancyhdr}

\pagestyle{fancy}

\fancyhf{}
\rhead{Computer Science 481}

\lhead{Software Project \assignmentnumber{}}

\rfoot{Page \thepage}
\lfoot{\duedate}

\usepackage{titlesec}
\titlespacing\section{0pt}{6pt plus 4pt minus 2pt}{4pt plus 2pt minus 2pt}

\newcommand{\labtitle}[1]
{
  \begin{center}
    \begin{center}
      \bf
      CMPSC 481\\Software Innovation II\\
      Spring 2019\\
      \medskip
    \end{center}
    \bf
    #1
  \end{center}
}

\begin{document}

\thispagestyle{empty}

\labtitle{Software Project \assignmentnumber{} \\ \assigneddate{} \\ \duedate{}}

\section*{Objectives}

To use GitHub and the GitHub flow model to collaboratively engineer, deliver,
and evaluate a software product.
%
Along with using GitHub features like the issue tracker and reviewing pull
requests, in this assignment you will use Markdown to complete technical writing
tasks and the Python programming language and many Python packages (e.g., Pytest
for automated testing) to create production-quality open-source programs for use
in an educational setting.
%
As a side effect of working in a team, you will also experience challenges
(e.g., the creation of merge conflicts in a version control repository) that
force you to develop practical solutions.
%
You will also gain experience in interacting with team members, technical
leaders, members of industry, and the course instructor.
%
Students will work together in a development team while mastering the technical
and professional skills in the field of software engineering, working towards
becoming a recognized software innovator who can design, implement, and release
software in production use at Allegheny College.
%
Finally, students will gain experience to distinguish themselves as software
innovators who can use systems such as Amazon Elastic Beanstalk and Amazon
Lambda for serverless computing.

\section*{Suggestions for Success}

\begin{itemize}
  \setlength{\itemsep}{0pt}

\item {\bf Use the laboratory computers}. The computers in the departmental
  laboratories feature specialized software for completing this course's
  assignments. If it is necessary for you to work on a different machine, be
  sure to regularly transfer your work to a laboratory machine so that you can
  check its correctness. If you cannot use a laboratory computer and you need
  help with the configuration of your own laptop, then please carefully explain
  its setup to a teaching assistant or the course instructor when you are asking
  questions.

% \item {\bf Regularly ask and answer questions}. Please log into Slack at the
%   start of a laboratory or practical session and then join the appropriate
%   channel. If you have a question about one of the steps in an assignment, then
%   you can post it to the designated channel. Or, you can ask a student sitting
%   next to you or talk with a teaching assistant or the course instructor.

% \item {\bf Store your files in GitHub}. As in all of your past assignments, you
%   will be responsible for storing all of your files (e.g., Python source code and
%   Markdown-based writing) in a Git repository using GitHub Classroom. Please
%   verify that you have saved your source code in your Git repository by using
%   \command{git status} to ensure that everything is updated. You can see if
%   your assignment submission meets the established correctness requirements by
%   using the provided checking tools on your local computer and in checking the
%   commits in GitHub.

% \item {\bf Keep all of your files}. Don't delete your programs, output files,
%   and written reports after you submit them through GitHub; you will need them
%   again when you study for the quizzes and examinations and work on the other
%   laboratory, practical, and final project assignments.

% \item {\bf Back up your files regularly}. All of your files are regularly
%   backed-up to the servers in the Department of Computer Science and, if you
%   commit your files regularly, stored on GitHub. However, you may want to use a
%   flash drive, Google Drive, or your favorite backup method to keep an extra
%   copy of your files on reserve. In the event of any type of system failure,
%   you are responsible for ensuring that you have access to a recent backup copy
%   of all your files.

\item {\bf Explore teamwork and technologies}. While certain aspects of these
  assignments will be challenging for you, each part is designed to give you the
  opportunity to learn both fundamental concepts in the field of computer
  science and explore advanced technologies that are commonly employed at a wide
  variety of companies. To explore and develop new ideas, you should regularly
  communicate with your team and/or the teaching assistants and tutors.

\item {\bf Hone your technical writing skills}. Computer science assignments
  require to you write technical documentation and descriptions of your
  experiences when completing each task. Take extra care to ensure that your
  writing is interesting and both grammatically and technically correct,
  remembering that computer scientists must effectively communicate and
  collaborate with their team members and the tutors, teaching assistants, and
  course instructor.

\item {\bf Review the Honor Code on the syllabus}. While you may discuss your
  assignments with others, copying source code or writing is a violation of
  Allegheny College's Honor Code.

\end{itemize}

\section*{Creating a Suite of Educational Software Tools}

% Accept the assignment

To access this assignment, you should go into the \channel{\#announcements}
channel in our Slack team and find the announcement that provides a link for it.
Copy this link and paste it into your web browser. Now, you should accept the
laboratory assignment and see that GitHub Classroom created a new GitHub
repository for you to access the assignment's starting materials and to store
the completed version of your assignment. Specifically, to access your new
GitHub repository for this assignment, please click the green ``Accept'' button
and then click the link that is prefaced with the label ``Your assignment has
been created here''. If you accepted the assignment and correctly followed these
steps, you should have created a GitHub repository with a name like
``Allegheny-Computer-Science-481-Spring-2019/computer-science-481-spring-2019-lab-2-gkapfham''.
%
Unless you provide the instructor with documentation of the extenuating
circumstances that you are facing, not accepting the assignment means that you
will receive a failing grade for it.
%
Instead of giving you access to ``starter'' code for this assignment, the
purpose of this GitHub repository is to facilitate the assessment of your own
mastery of the professional and technical skills in software innovation.

% Details of what students must complete

This assignment asks you to work in a entire-class team to create a suite of
educational software tools that for use by the students and faculty in the
Department of Computer Science at Allegheny College. Your goal for this project
is to implement a suite of useful, production-quality software tools, deployed
as either web sites or command-line applications that can be used starting in
the Fall 2019 semester.
%
For this assignment, you will collaborate with the members of your class to
implement these tools in the Python programming language, leveraging, for
instance, either the Django web application framework or the Chalice framework
for creating and deploying Lambda functions on the Amazon serverless computing
cloud.
%
Your development team should organize itself into sub-teams that will focus on
implementing a total of five educational software tools.
%
Then, each sub-team should pick a project, decide on a name for that project,
create a GitHub repository in the GatorEducator GitHub organization, and start
using the issue tracker to identify the key projects on which you will focus.
%
Every team member is responsible for working together to handle these issues and
any others that arise during the completion of this long-term software project.

Using this list as a starting point, the sub-teams should each pick a distinct
software project:

\vspace*{-.1in}

\begin{enumerate}
  \setlength{\itemsep}{0pt}

  \item {\bf Course Survey}: A survey that students can complete to share, for
    instance, their concerns about course content and instruction. This survey
    should be customizable, thereby allowing an instructor to ask new questions
    of students on a regular basis. The survey should support a wide range of
    question types and be available through the terminal window or a web
    interface.

  \item {\bf Academic Advising}: A means by which students can effectively
    communicate with their academic adviser, supporting, for instance, the
    submission of a status update and a four-year course plan. It would also
    allow faculty to record that they have met with their advisees.

  \item {\bf Interactive Quiz}: A customizable quiz system that allows faculty
    to define questions, administer a quiz, and receive a copy of results in a
    fashion suitable for semi-automated grading.

  \item {\bf Student Petition}: A system that will allow students to upload a
    petition for a change in, for instance, their graduation requirements. Along
    with giving faculty a way to vote on the petition, this system should then
    automatically notify students of the result of the vote.

  \item {\bf Review Board Checklist}: A system that will allow students to upload a


\end{enumerate}

\section*{Collaborating with Your Software Engineering Team}

Your team should use GitHub and its features (e.g., issue tracker, pull
requests, commit log, and code review request) to complete all of the tasks
referenced in the previous section.
%
Aiming to manage risk and estimate the effort required for individual team
members to complete this project, you should assign people to teams, roles, and
tasks. While it is acceptable for you to have in-person discussions with your
team members or to talk about the project through Slack, please remember that
all important discussions and decisions must be documented through GitHub.
Finally, as you are working with your team, you should carefully document your
experiences and contributions so that you can share them through writing stored
in the repository created by GitHub Classroom.

Since multiple approaches may support the effective completion of the required
software, this assignment does not dictate team organization or communication
strategies. The students in the course should instead work with each other, the
team leaders, and the course instructor to identify team roles and strategies
for effective organization and communication. With that said, you should plan to
use either forks or branches of a GitHub repository to organize your work.
%
Once a specific branch/fork contains the finished version of its associated
deliverable, a team member should create a pull request for discussion and
review. If the team leaders, the technical leaders, and the course instructor
judge that the pull request has all of the expected characteristics, then it
should be merged into the ``master'' branch of the appropriate repository. If
the pull request is not accepted, then team member(s) should improve it until it
meets every reviewer's expectations. Your team should continue to use this
model, called ``GitHub flow'', to support the completion of all deliverables.
%
Students with questions about the use of GitHub should first talk to a team
leader.

\section*{Self Evaluation and Project Retrospective}

% Evaluation and retrospective

Your GitHub repository for this assignment contains a Markdown file that you
will use to document, evaluate, and reflect on your contributions to this
project.
%
Your evaluation of your own work should focus on your mastery of the technical
and professional skills that are necessary to become a software innovator. You
should thoughtfully reflect on your current areas of expertise and opportunities
for improvement. As you work on this project, you must be proactive in finding
ways to master skills such as using version control, documenting and refactoring
source code, adding and testing new source code modules, and writing technical
documentation. In addition to writing a publicly available blog post that
documents and reflects on your experiences, your private reflection should
analyze the challenges that you faced and the strategies that you adopted to
overcome them.

% Checking of the assessment sheet and the code of conduct guide

Please remember that Travis CI is configured to use both \command{mdl} and
\command{proselint} to check the Markdown files in your repository created by
GitHub Classroom.
%
If you saved the files correctly and your writing meets all of the requirements
set by these linting tools, then you will see a green \checkmark{} in the
listing of commits in GitHub after awhile. If your submission does not meet the
requirements, a red \naughtmark{} will appear instead. The instructor will
reduce a student's grade for this assignment if the red \naughtmark{} appears on
the last commit in GitHub immediately before the assignment's due date. Yet, if
the green \checkmark{} appears on the last commit in your GitHub repository,
then you satisfied the basic linting checks for the Markdown file that contains
your work assessment and reflection.

\section*{Suggested Schedule for the Software Project}

The course instructor invites the students in this class to work together to
devise a schedule by which they can complete the software product by the stated
deadline. Overall, you will work on this assignment for six weeks. Here is a
suggestion for a schedule to complete GatorGrader:

\begin{itemize}

  \setlength{\itemsep}{0pt}

  \item {\bf March 28}: As you start this project, please set a regular
    out-of-class meeting schedule and make arrangements to regularly report your
    progress to both your team leader and the course instructor. As you develop
    your ideas, make sure that you can carefully leverage GitHub's features to
    ensure that there is no duplicative work and that collaboration proceeds
    smoothly. Please see the instructor if you have questions about picking the
    project on which your sub-team will focus as you completes this suite of
    innovative educational software.
    %
    You should develop a list of features for your program and explore what
    technologies you will use to support its implementation, focusing on
    frameworks supported by the Python language.

  \item {\bf April 4}: Once you have finished, for instance, organizing your
    teams and raising issues in GitHub, further explore the Python language
    features and packages you will need to implement your features. You should
    start to implement a prototype of your assigned features or a start on your
    defect fixes. Please see the instructor and your team leader if you are
    facing severe challenges that may prevent you from finishing a prototype
    during this week.

  \item {\bf April 11}: Finish implementing a prototype of your tool and give
    a demonstration to the instructors and students who will use your tool,
    requesting feedback that you incorporate into subsequent revisions. Your aim
    should be to finish all implementation and debugging tasks, ultimately
    leading to a full-featured demonstration that surfaces detailed feedback.

  \item {\bf April 18}: Finish implementing and debugging all key features and
    complete all major defect fixes, ultimately leading to a full-featured
    demonstration and further feedback from additional people who use the tool.
    At this stage, you should have demonstrated your tool to multiple students
    and faculty at Allegheny, helping them to use the tool in their own work.

  \item {\bf April 25}: While continuing to incrementally enhance your tool,
    release a completed version so that external individuals can use it and
    provide detailed feedback through the issue tracker. Finish your tool
    documentation and reflection and publish a draft of your blog post so that
    it is available for peer review by the instructor and your colleagues in
    this course.

  \item {\bf Mary 2}: Release a production quality tool suitable for use by
    other students and by all of the instructors at Allegheny College. Publish
    the final version of your technical blog post and submit your private
    reflection through your repository created by GitHub Classroom.

\end{itemize}

\noindent After project completion you will receive a letter grade for your
cumulative work on this project. During the completion of this project you will
also receive feedback and advice from the instructor.

\section*{Summary of the Required Deliverables}

\noindent Students do not need to submit printed source code or technical
writing for any assignment in this course. Instead, this assignment invites you
to submit, using GitHub, the following deliverables.
%
Unless you provide the instructor with documentation of the severe and
extenuating circumstances that you are facing, no late work will be considered
towards your grade for this software project.
%
Using the GitHub repository for the GatorGrader project, the instructor will, as
appropriate, share regular public feedback with all of the team members. All
students will privately receive a letter grade for each of the following aspects
of their contribution to this long-term software project.

\begin{enumerate}

\setlength{\itemsep}{0in}

\item Available for download from
  \url{https://github.com/GatorEducator/gatorgrouper}, an enhanced and
  high-quality automated grading tool that is written in Python and in
  production use.

\item Submitted through your repository created by GitHub Classroom, a self
  evaluation that documents what you learned, the challenges that you faced, and
  the contributions that you made to the GatorGrader project. Along with
  explaining how you interacted with non-team members (e.g., experts who
  provided feedback on the tool), this evaluation should also detail the ways in
  which you helped teachers and students learn how to use the GatorGrader tool.

\item Published on your web site, a publicly available blog post that overviews
  the GatorGrader tool, documents the tasks that you completed, and reflects on
  your experiences during the enhancement of this tool. In addition to linking
  to GatorGrader's GitHub repository and your contributions, your blog post
  should use source code and command-line examples to give a detailed
  introduction to how GatorGrader automatically checks code and technical
  writing.

\end{enumerate}

\end{document}
